\documentclass{beamer}

\usepackage[T1,T2A]{fontenc}
\usepackage[utf8]{inputenc}
\usepackage[english,russian]{babel}
\usepackage{tempora}

\usepackage[auto]{contour}
\usepackage{color}

\definecolor{iu9dept}{RGB}{40,40,60} %это нужно, чтобы ввести имя кафедры на нескольких строчках
\definecolor{iu9shade}{RGB}{245,245,255} % поскольку многострочных contour-объектов не бывает

\newcommand{\iutitle}[1]{\contour[10]{iu9shade}{\textcolor{iu9dept}{#1}}}
\newcommand{\iutitleupd}[1]{\contour[10]{iu9shade!90!gray}{\textcolor{iu9dept}{#1}}}

\usetheme{IU9advanced}

\title[Подпись в футере]{Заголовок доклада из одной строчки}%Заголовок - с контуром, из одной строчки
\subtitle{и подзаголовок}%Подзаголовок - то же, но без контура. Если хочется сделать одного размера - используем тег \LARGE
\date[Дата в футере]{Название конференции и дата} %Я не вывожу название конференции на слайд, потому что её логично поместить вместо названия кафедры для внешних конференций, или под чертой с эмблемой - для внутренних.
\author[Сокращение, не идёт в футер]{Имя автора \\\texttt{и@почта}}
\supervisor{Имя руководителя или автора-2\\\texttt{и@почта}}
\institute{\contourlength{0.05em}\iutitle{МГТУ им. Н.Э. Баумана}\\\vspace*{-2ex}\\ \iutitle{Кафедра <<Теоретическая информатика}\\
\iutitleupd{\qquad\qquad\qquad\qquad и компьютерные технологии>>}}%если конференция внешняя, здесь лучше поставить название конференции, а кафедру упоминать не нужно

\setbeamertemplate{blocks}[rounded][shadow=true]

\begin{document}

\begin{frame}
\titlepage
\end{frame}

\setbeamertemplate{background}[iunine] %Это фон чертой под заголовком слайда
\setbeamertemplate{block begin}[iunine] %Начало блока с двойным отчёркиванием
\setbeamertemplate{block end}[iunine] %Конец блока с штриховым отчёркиванием

\begin{frame}{Стандартный стиль}
\begin{block}{Заголовок блока в стиле IU9}
\begin{itemize}
\item Item1
\item Item2
\end{itemize}
\end{block}%отчёркивание на конце блока

\begin{block}{}
Блок с пустым заголовком в стиле IU9. Отчёркивание никуда не делось.
\end{block}%отчёркивание на конце блока
\end{frame}

\setbeamertemplate{background}[notitle] %Это фон только с футером
\setbeamertemplate{block begin}[simple] %Начало блока без выделения
\setbeamertemplate{block end}[simple] %Конец блока без выделения

\begin{frame}{}
Стиль чистого листа. Фоновый стиль такого типа удобно использовать для слайдов без заголовка.
\begin{block}{Заголовок блока в стиле чистого листа}
\begin{itemize}
\item Item1
\item Item2
\end{itemize}
\end{block}%отчёркивание на конце блока

\begin{block}{}
Блок с пустым заголовком в стиле чистого листа. Не виден.
\end{block}%отчёркивание на конце блока

\end{frame}

\setbeamertemplate{background}[iunine] %Это фон чертой под заголовком слайда
\setbeamertemplate{block begin}[iunineone] %Начало блока с одинарным отчёркиванием
\setbeamertemplate{block end}[default] %Конец блока с штриховым отчёркиванием

\begin{frame}{Комбинированный стиль}
\begin{block}{}
Блок с пустым заголовком с одинарным отчёркиванием. Отчёркивание в конце отключено.
\end{block}%отчёркивание на конце блока

\setbeamertemplate{block end}[iunine] %Конец блока с штриховым отчёркиванием

\begin{block}{}
Блок с пустым заголовком с одинарным отчёркиванием в начале и отчёркиванием в конце.
\end{block}%отчёркивание на конце блока
\end{frame}

\setbeamertemplate{background}[notitle] %Это фон только с футером
\setbeamertemplate{block begin}[simple] %Начало блока без выделения
\setbeamertemplate{block end}[simple] %Конец блока без выделения
\end{document}
